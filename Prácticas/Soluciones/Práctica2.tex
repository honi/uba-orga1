\documentclass{article}

\usepackage[utf8]{inputenc}
\usepackage[spanish]{babel} % Agregar es-nodecimaldot si hay problemas con el separador de decimales y de miles.
\usepackage{amsmath} % Agregar cuando se necesiten: amscd, amssymb, amsthm, latexsym
\usepackage{enumerate}
\usepackage{graphicx}
\usepackage{multicol}
\usepackage{svg}

% Tamaño de hoja y márgenes
\usepackage[a4paper,top=1cm,bottom=2cm,left=1cm,right=1cm]{geometry}
\setlength{\parindent}{0em}
\setlength{\parskip}{1em}
\setlength{\intextsep}{1em}
\setlength{\abovedisplayskip}{1em}
\setlength{\belowdisplayskip}{1em}
\setlength{\abovedisplayshortskip}{1em}
\setlength{\belowdisplayshortskip}{1em}

% Numeración de las secciones y el índice
\setcounter{tocdepth}{1}
\renewcommand{\thesubsection}{\thesection.\alph{subsection}}

% Color de los links externos
\usepackage{hyperref}
\hypersetup{colorlinks=true, linkcolor=black, urlcolor=blue}

% Permite math mode adentro de listings (bloques de "código")
\usepackage{listings}
\lstset{basicstyle=\ttfamily, mathescape}

% Alias para íconos/símbolos
\usepackage{pifont}
\newcommand{\xmark}{\color{purple}\ding{54}}

% Otros alias
\newcommand{\xor}{\oplus}
\newcommand{\nor}{\downarrow}

% Columna "x" tiene fondo amarillo
\usepackage{colortbl}
\newcolumntype{x}{>{\columncolor[HTML]{FFF2CC}}c}


\title{Organización del Computador 1}
\author{Práctica 2}
\date{1er cuatrimestre 2022}

\begin{document}

\maketitle
\tableofcontents
\newpage

\section{Ejercicio 1}

Calculando las tablas de verdad podemos ver la equivalencia de las fórmulas booleanas.

\subsection{}

$p=(p.q)+(p.\overline{q})$

\begin{tabular}{|x|c|c|x|c|}
    $p$ & $q$ & $(p.q)$ & $+$ & $(p.\overline{q})$ \\
    \hline
    0 & 0 & 0 & 0 & 0 \\
    0 & 1 & 0 & 0 & 0 \\
    1 & 0 & 0 & 1 & 1 \\
    1 & 1 & 1 & 1 & 0 \\
\end{tabular}

\subsection{}

$x.z = (x+y).(x+\overline{y}).(\overline{x}+z)$

\begin{tabular}{|c|c|c||x|c|c|c|x|c|}
    $x$ & $y$ & $z$ & $x.z$ & $(x+y)$ & $.$ & $(x+\overline{y})$ & $.$ & $(\overline{x}+z)$ \\
    \hline
    0 & 0 & 0 & 0 & 0 & 0 & 1 & 0 & 1 \\
    0 & 0 & 1 & 0 & 0 & 0 & 1 & 0 & 1 \\
    0 & 1 & 0 & 0 & 1 & 0 & 0 & 0 & 1 \\
    0 & 1 & 1 & 0 & 1 & 0 & 0 & 0 & 1 \\
    1 & 0 & 0 & 0 & 1 & 1 & 1 & 0 & 0 \\
    1 & 0 & 1 & 1 & 1 & 1 & 1 & 1 & 1 \\
    1 & 1 & 0 & 0 & 1 & 1 & 1 & 0 & 0 \\
    1 & 1 & 1 & 1 & 1 & 1 & 1 & 1 & 1 \\
\end{tabular}

\section{Ejercicio 2}

Resolviendo mediante propiedades llegamos a 2 fórmulas que a priori no parecen ser equivalentes. Notar en la última línea que a la izquierda tenemos $\overline{y.z}$ mientras que a la derecha tenemos $\overline{y}.\overline{z}$.
\begin{flalign*}
    x \xor (y.z) &= (x \xor y).(x \xor z) &&\\
    \overline{x}.y.z + x.\overline{y.z} &= (\overline{x}.y + x.\overline{y}).(\overline{x}.z + x.\overline{z}) &&\\
    \overline{x}.y.z + x.\overline{y.z} &= (\overline{x}.y + x.\overline{y}).\overline{x}.z + (\overline{x}.y + x.\overline{y}).x.\overline{z} &&\\
    \overline{x}.y.z + x.\overline{y.z} &= \overline{x}.y.\overline{x}.z + x.\overline{y}.\overline{x}.z + \overline{x}.y.x.\overline{z} + x.\overline{y}.x.\overline{z} &&\\
    \overline{x}.y.z + x.\overline{y.z} &= \overline{x}.y.z + x.\overline{y}.\overline{z}
\end{flalign*}

Calculamos la tabla de verdad para verificar.

\begin{tabular}{|c|c|c||c|x|c|c|x|c|c|}
    $x$ & $y$ & $z$ & $x$ & $\xor$ & $(y.z)$ & $(x \xor y)$ & . & $(x \xor z)$ & \\
    \hline
    1 & 1 & 1 & 1 & 0 & 1 & 0 & 0 & 0 &  \\
    1 & 1 & 0 & 1 & 1 & 0 & 0 & 0 & 1 & \xmark \\
    1 & 0 & 1 & 1 & 1 & 0 & 1 & 0 & 0 & \xmark \\
    1 & 0 & 0 & 1 & 1 & 0 & 1 & 1 & 1 &  \\
    0 & 1 & 1 & 0 & 1 & 1 & 1 & 1 & 1 &  \\
    0 & 1 & 0 & 0 & 0 & 0 & 1 & 0 & 0 &  \\
    0 & 0 & 1 & 0 & 0 & 0 & 0 & 0 & 1 &  \\
    0 & 0 & 0 & 0 & 0 & 0 & 0 & 0 & 0 &  \\
\end{tabular}

Conclusión: la propiedad planteada es falsa.

\section{Ejercicio 3}

\subsection{}

Verdadero, con el operador NAND ($p|q = \overline{p.q}$) podemos representar todas las funciones booleanas: AND, OR, NOT.

Recordemos la tabla de verdad del NAND.

\begin{tabular}{|c|c|c|x|}
    $p$ & $q$ & $p.q$ & $p|q = \overline{p.q}$ \\
    \hline
    0 & 0 & 0 & 1 \\
    0 & 1 & 0 & 1 \\
    1 & 0 & 0 & 1 \\
    1 & 1 & 1 & 0 \\
\end{tabular}

\subsubsection{NOT}

Utilizando la misma entrada 2 veces en un NAND podemos obtener un NOT.

$p|p = \overline{p}$

\begin{tabular}{|c|x|x|}
    $p$ & $p|p$ & $\overline{p}$ \\
    \hline
    0 & 1 & 1 \\
    1 & 0 & 0 \\
\end{tabular}

\begin{figure}[ht]
    \includesvg{Circuitos/NAND_NOT}
\end{figure}

\subsubsection{AND}

Utilizando el NOT ya construido, podemos encadenarlo a la salida de un NAND para cancelar su negación y así obtener el resultado original del AND.

$(p|q)|(p|q) = p.q$

\begin{tabular}{|c|c|c|x|x|}
    $p$ & $q$ & $p|q$ & $(p|q)|(p|q)$ & $p.q$ \\
    \hline
    0 & 0 & 1 & 0 & 0 \\
    0 & 1 & 1 & 0 & 0 \\
    1 & 0 & 1 & 0 & 0 \\
    1 & 1 & 0 & 1 & 1 \\
\end{tabular}

\begin{figure}[ht]
    \includesvg{Circuitos/NAND_AND}
\end{figure}

\subsubsection{OR}

$(p|p)|(q|q) = \overline{p}|\overline{q} = \overline{\overline{p}.\overline{q}} = \overline{\overline{p}}+\overline{\overline{q}} = p+q$

\begin{tabular}{|c|c|c|x|c|x|}
    $p$ & $q$ & $(p|p)$ & $|$ & $(q|q)$ & $p+q$ \\
    \hline
    0 & 0 & 1 & 0 & 1 & 0 \\
    0 & 1 & 1 & 1 & 0 & 1 \\
    1 & 0 & 0 & 1 & 1 & 1 \\
    1 & 1 & 0 & 1 & 0 & 1 \\
\end{tabular}

\begin{figure}[ht]
    \includesvg{Circuitos/NAND_OR}
\end{figure}

\subsection{}

Verdadero, con el operador NOR ($p\downarrow q = \overline{p+q}$) podemos representar todas las funciones booleanas: AND, OR, NOT.

Recordemos la tabla de verdad del NOR.

\begin{tabular}{|c|c|c|x|}
    $p$ & $q$ & $p+q$ & $p\downarrow q = \overline{p+q}$ \\
    \hline
    0 & 0 & 0 & 1 \\
    0 & 1 & 1 & 0 \\
    1 & 0 & 1 & 0 \\
    1 & 1 & 1 & 0 \\
\end{tabular}

Observemos que los circuitos para armar las funciones booleanas utilizando solo la compuerta NOR son análogos a los utilizandos con la compuerta NAND.

\subsubsection{NOT}

$p\downarrow p = \overline{p}$

\begin{tabular}{|c|x|x|}
    $p$ & $p\downarrow p$ & $\overline{p}$ \\
    \hline
    0 & 1 & 1 \\
    1 & 0 & 0 \\
\end{tabular}

\begin{figure}[ht]
    \includesvg{Circuitos/NOR_NOT}
\end{figure}

\subsubsection{AND}

$(p\downarrow p)\downarrow(q\downarrow q) = \overline{p}\downarrow\overline{q} = \overline{\overline{p}+\overline{q}} = \overline{\overline{p}}.\overline{\overline{q}} = p.q$

\begin{tabular}{|c|c|c|x|c|x|}
    $p$ & $q$ & $(p\downarrow p)$ & $\downarrow$ & $(q\downarrow q)$ & $p.q$ \\
    \hline
    0 & 0 & 1 & 0 & 1 & 0 \\
    0 & 1 & 1 & 0 & 0 & 0 \\
    1 & 0 & 0 & 0 & 1 & 0 \\
    1 & 1 & 0 & 1 & 0 & 1 \\
\end{tabular}

\begin{figure}[ht]
    \includesvg{Circuitos/NOR_AND}
\end{figure}

\subsubsection{OR}

$(p\downarrow q)\downarrow(p\downarrow q) = (\overline{p+q}) \downarrow (\overline{p+q}) = \overline{\overline{p+q}} = p+q$

\begin{tabular}{|c|c|c|x|x|}
    $p$ & $q$ & $p\downarrow q$ & $(p\downarrow q)\downarrow(p\downarrow q)$ & $p+q$ \\
    \hline
    0 & 0 & 1 & 0 & 0 \\
    0 & 1 & 0 & 1 & 1 \\
    1 & 0 & 0 & 1 & 1 \\
    1 & 1 & 0 & 1 & 1 \\
\end{tabular}

\begin{figure}[ht]
    \includesvg{Circuitos/NOR_OR}
\end{figure}

\section{Ejercicio 4}

Resuelto en el ejercicio 3.

\section{Ejercicio 5}

\begin{tabular}{|c|c|c|x|}
    $p$ & $q$ & $r$ & $p.q.r$ \\
    \hline
    0 & 0 & 0 & 0 \\
    0 & 0 & 1 & 0 \\
    0 & 1 & 0 & 0 \\
    0 & 1 & 1 & 0 \\
    1 & 0 & 0 & 0 \\
    1 & 0 & 1 & 0 \\
    1 & 1 & 0 & 0 \\
    1 & 1 & 1 & 1 \\
\end{tabular}

\begin{figure}[ht]
    \includesvg{Circuitos/P2_EJ5}
\end{figure}

\section{Ejercicio 6}

\begin{tabular}{|c|c|c|x|}
    $A$ & $B$ & $C$ & $F(A,B,C)$ \\
    \hline
    0 & 0 & 0 & 0 \\
    0 & 0 & 1 & 0 \\
    0 & 1 & 0 & 0 \\
    0 & 1 & 1 & 1 \\
    1 & 0 & 0 & 1 \\
    1 & 0 & 1 & 1 \\
    1 & 1 & 0 & 0 \\
    1 & 1 & 1 & 1 \\
\end{tabular}

\emph{Nota: La tabla de verdad fue ordenada para una lectura más fácil.}

\subsection{}

$F(A,B,C) = \overline{A}.B.C + A.\overline{B}.\overline{C} + A.\overline{B}.C + A.B.C$

La implementación literal requiere un total de 15 compuertas: 3 OR, 8 AND y 4 NOT.

\subsection{}

$F(A,B,C) = \overline{A}.B.C + A.\overline{B}.\overline{C} + A.\overline{B}.C + A.B.C = B.C(A + \overline{A}) + A.\overline{B}(\overline{C} + C) = B.C + A.\overline{B}$

La implementación optimizada requiere un total de 4 compuertas: 1 OR, 2 AND y 1 NOT.

\begin{tabular}{|c|c|c||c|x|c|}
    $A$ & $B$ & $C$ & $B.C$ & $+$ & $A.\overline{B}$ \\
    \hline
    0 & 0 & 0 & 0 & 0 & 0 \\
    0 & 0 & 1 & 0 & 0 & 0 \\
    0 & 1 & 0 & 0 & 0 & 0 \\
    0 & 1 & 1 & 1 & 1 & 0 \\
    1 & 0 & 0 & 0 & 1 & 1 \\
    1 & 0 & 1 & 0 & 1 & 1 \\
    1 & 1 & 0 & 0 & 0 & 0 \\
    1 & 1 & 1 & 1 & 1 & 0 \\
\end{tabular}

\begin{figure}[ht]
    \includesvg{Circuitos/P2_EJ6}
\end{figure}

\pagebreak

\section{Ejercicio 7: demultiplexor}

En la primer tabla de verdad planteamos una columna para los valores explícitos de $e_0$. En la segunda tabla eliminamos esta columna y en cambio planteamos $e_0$ como el valor de salida cuando las líneas de control computan a 1 para esa fila. Es decir, dada la fórmula booleana para cada salida, primero vemos si el AND entre las 2 líneas de control daría 1, y solo en ese caso el resultado final va a ser determinado por $e_0$. En todos los otros casos, como el AND entre las líneas de control da 0, es indistinto el valor de $e_0$ y la salida va a ser siempre $0$.

\begin{multicols}{2}
    \begin{tabular}{|c|c|c||c|c|c|c|}
        $c_1$ & $c_0$ & $e_0$ & $s_3$ & $s_2$ & $s_1$ & $s_0$ \\
        \hline
        0 & 0 & 0 & 0 & 0 & 0 & 0 \\
        0 & 1 & 0 & 0 & 0 & 0 & 0 \\
        1 & 0 & 0 & 0 & 0 & 0 & 0 \\
        1 & 1 & 0 & 0 & 0 & 0 & 0 \\
        0 & 0 & 1 & 0 & 0 & 0 & 1 \\
        0 & 1 & 1 & 0 & 0 & 1 & 0 \\
        1 & 0 & 1 & 0 & 1 & 0 & 0 \\
        1 & 1 & 1 & 1 & 0 & 0 & 0 \\
    \end{tabular}

    \begin{tabular}{|c|c||c|c|c|c|}
        $c_1$ & $c_0$ & $s_3$ & $s_2$ & $s_1$ & $s_0$ \\
        \hline
        0 & 0 & 0 & 0 & 0 & $e_0$ \\
        0 & 1 & 0 & 0 & $e_0$ & 0 \\
        1 & 0 & 0 & $e_0$ & 0 & 0 \\
        1 & 1 & $e_0$ & 0 & 0 & 0 \\
    \end{tabular}
\end{multicols}

Podemos plantear las siguientes fórmulas booleanas para cada una de las salidas.

$
s_3 = c_1.c_0.e_0
\hspace{2em}
s_2 = c_1.\overline{c_0}.e_0
\hspace{2em}
s_1 = \overline{c_1}.c_0.e_0
\hspace{2em}
s_0 = \overline{c_1}.\overline{c_0}.e_0
$

\begin{figure}[ht]
    \includesvg{Circuitos/Demultiplexor_v1}
\end{figure}

\pagebreak

\section{Ejercicio 8: codificador}

\subsection{}

\begin{tabular}{|c|c|c|c||c|c|}
    $e_0$ & $e_1$ & $e_2$ & $e_3$ & $s_1$ & $s_0$ \\
    \hline
    1 & 0 & 0 & 0 & 0 & 0 \\
    0 & 1 & 0 & 0 & 0 & 1 \\
    0 & 0 & 1 & 0 & 1 & 0 \\
    0 & 0 & 0 & 1 & 1 & 1 \\
\end{tabular}

$s_1 = \overline{e_0}.\overline{e_1}.e_2.\overline{e_3} + \overline{e_0}.\overline{e_1}.\overline{e_2}.e_3$

$s_0 = \overline{e_0}.e_1.\overline{e_2}.\overline{e_3} + \overline{e_0}.\overline{e_1}.\overline{e_2}.e_3$

\emph{Nota: Consideramos $s_1$ como el bit más significativo de la representación binaria del número $i$.}

\subsection{}

Para determinar si el estado de la entrada es válido necesitamos que $v=1$ únicamente cuando hay una sola entrada en $1$.

\begin{tabular}{|c|c|c|c||c|}
    $e_0$ & $e_1$ & $e_2$ & $e_3$ & v \\
    \hline
    0 & 0 & 0 & 0 & 0 \\
    0 & 0 & 0 & 1 & 1 \\
    0 & 0 & 1 & 0 & 1 \\
    0 & 0 & 1 & 1 & 0 \\
    0 & 1 & 0 & 0 & 1 \\
    0 & 1 & 0 & 1 & 0 \\
    0 & 1 & 1 & 0 & 0 \\
    0 & 1 & 1 & 1 & 0 \\
    1 & 0 & 0 & 0 & 1 \\
    1 & 0 & 0 & 1 & 0 \\
    1 & 0 & 1 & 0 & 0 \\
    1 & 0 & 1 & 1 & 0 \\
    1 & 1 & 0 & 0 & 0 \\
    1 & 1 & 0 & 1 & 0 \\
    1 & 1 & 1 & 0 & 0 \\
    1 & 1 & 1 & 1 & 0 \\
\end{tabular}

Traducimos la tabla de verdad a una fórmula booleana literal y simplificamos usando propiedades.
\begin{flalign*}
v &= e_0 . \overline{e_1} . \overline{e_2} . \overline{e_3} + \overline{e_0} . e_1 . \overline{e_2} . \overline{e_3} + \overline{e_0} . \overline{e_1} . e_2 . \overline{e_3} + \overline{e_0} . \overline{e_1} .\overline{e_2} . e_3 &&\\
&= (e_0 . \overline{e_1} + \overline{e_0} . e_1) . \overline{e_2} . \overline{e_3} + \overline{e_0} . \overline{e_1} . (e_2 . \overline{e_3} + \overline{e_2} . e_3) &&\\
&= (e_0 \xor e_1) . (e_2 \nor e_3) + (e_0 \nor e_1) . (e_2 \xor e_3)
\end{flalign*}

La fórmula se puede describir de la siguiente forma: la salida $v$ será 1 si $e_0$ o $e_1$ son 1 (pero no ambos al mismo tiempo) y a su vez $e_2$ y $e_3$ son ambos 0. O, de forma análoga, si $e_2$ o $e_3$ son 1 (pero no ambos al mismo tiempo) y a su vez $e_0$ y $e_1$ son ambos 0.

\pagebreak

Planteamos un único circuito para el codificador con la salida que indica si la entrada está en un estado válido.

\begin{figure}[ht]
    \includesvg{Circuitos/Codificador_v2}
\end{figure}

\pagebreak

\section{Ejercicio 9: decodificador}

\subsection{}

\begin{tabular}{|c|c||c|c|c|c|}
    $e_1$ & $e_0$ & $s_3$ & $s_2$ & $s_1$ & $s_0$ \\
    \hline
    0 & 0 & 0 & 0 & 0 & 1 \\
    0 & 1 & 0 & 0 & 1 & 0 \\
    1 & 0 & 0 & 1 & 0 & 0 \\
    1 & 1 & 1 & 0 & 0 & 0 \\
\end{tabular}

$
s_3 = e_1.e_0
\hspace{2em}
s_2 = e_1.\overline{e_0}
\hspace{2em}
s_1 = \overline{e_1}.e_0
\hspace{2em}
s_0 = \overline{e_1}.\overline{e_0}
$

\begin{figure}[ht]
    \includesvg{Circuitos/Decodificador}
\end{figure}

\subsection{}

\begin{figure}[ht]
    \includesvg{Circuitos/Demultiplexor_v2}
\end{figure}

\pagebreak

\section{Ejercicio 10: carry left shifter 3-4}

\subsection{}

$
s_3 = c_0 . e_2
\hspace{2em}
s_2 = \overline{c_0} . e_2 + c_0 . e_1
\hspace{2em}
s_1 = \overline{c_0} . e_1 + c_0 . e_0
\hspace{2em}
s_0 = \overline{c_0} . e_0
$

\begin{figure}[ht]
    \includesvg{Circuitos/CarryLeftShifter}
\end{figure}

\subsection{}

El shift a la izquierda equivale a multiplicar por 2. El shift a la derecha equivale a la división entera por 2 (es decir, dividir por 2 redondeado hacia abajo).

\begin{lstlisting}
0110$_2$ = 6$_{10}$
0110$_2$ $\ll$ 1 = 1100$_2$ = 12$_{10}$
0110$_2$ $\gg$ 1 = 0011$_2$ = 3$_{10}$
\end{lstlisting}

\begin{lstlisting}
0101$_2$ = 5$_{10}$
0101$_2$ $\ll$ 1 = 1010$_2$ = 10$_{10}$
0101$_2$ $\gg$ 1 = 0010$_2$ = 2$_{10}$
\end{lstlisting}

\pagebreak

\section{Ejercicio 11: full adder}

\subsection{}

\begin{multicols}{2}
    \begin{flalign*}
        \text{suma}
        &= \overline{A} . \overline{B} . C_{in} + \overline{A} . B . \overline{C_{in}} + A . \overline{B} . \overline{C_{in}} + A . B . C_{in} &&\\
        &= (\overline{A} . \overline{B} + A . B) . C_{in} + (\overline{A} . B + A . \overline{B}) . \overline{C_{in}} &&\\
        &= (\overline{A \xor B}) . C_{in} + (A \xor B) . \overline{C_{in}} &&\\
        &= (A \xor B) \xor C_{in}
    \end{flalign*}

    \begin{flalign*}
        C_{out}
        &= \overline{A} . B . C_{in} + A . \overline{B} . C_{in} + A . B . \overline{C_{in}} + A . B . C_{in} &&\\
        &= (\overline{A} . B + A . \overline{B}) . C_{in} + (\overline{C_{in}} + C_{in}) . A . B &&\\
        &= (A \xor B) . C_{in} + A . B
    \end{flalign*}
\end{multicols}

\begin{figure}[ht]
    \includesvg{Circuitos/FullAdder}
\end{figure}

\subsection{}

El retardo total del circuito será $max\{ \text{retardo}(\text{suma}), \text{retardo}(C_{out}) \}$. La salida suma utiliza 2 compuertas mientras que $C_{out}$ utiliza 4. Nos concentramos entonces en ver el retardo de $C_{out}$ para cada instante.

\begin{enumerate}
    \item Se activan las compuertas para las operaciones: $A \xor B$ y $A . B$
    \item Se activa la compuerta para la operación AND: $(A \xor B) \boldsymbol{.} C_{in}$
    \item Se activa la compuerta para la operación OR: $(A \xor B) . C_{in} \boldsymbol{+} A . B$
\end{enumerate}


Por lo tanto, el retardo total del circuito para producir todas su señales de salida es de $3t$.

\end{document}
