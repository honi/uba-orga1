\input{Algo1Macros}
\usepackage{caratula}
\usepackage{enumerate}
\usepackage{hyperref}
\usepackage{xcolor,colortbl}
\usepackage{svg}
\usepackage{pifont}

\setlength{\parindent}{0em}
\setlength{\parskip}{1em}
\setlength{\intextsep}{1em}
\setlength{\abovedisplayskip}{1em}
\setlength{\belowdisplayskip}{1em}
\setlength{\abovedisplayshortskip}{1em}
\setlength{\belowdisplayshortskip}{1em}
\decimalpoint
\setcounter{tocdepth}{3}
\hypersetup{colorlinks=true, linkcolor=black, urlcolor=blue}
\newcolumntype{x}{>{\columncolor[HTML]{FFF2CC}}c}
\renewcommand{\thesubsection}{\thesection.\alph{subsection}}
\newcommand{\xor}{\oplus}
\newcommand{\xmark}{\color{purple}\ding{54}}

\begin{document}

\titulo{Práctica 2}
\subtitulo{Lógica Digital}
\fecha{1er cuatrimestre 2022}
\materia{Organización del Computador 1}
\integrante{Jonathan Bekenstein}{348/11}{jbekenstein@dc.uba.ar}

\maketitle
\tableofcontents
\newpage

\section{Ejercicio 1}

Calculando las tablas de verdad podemos ver la equivalencia de las fórmulas booleanas.

\subsection{}

$p=(p.q)+(p.\overline{q})$

\begin{tabular}{|x|c|c|x|c|}
    $p$ & $q$ & $(p.q)$ & $+$ & $(p.\overline{q})$ \\
    \hline
    0 & 0 & 0 & 0 & 0 \\
    0 & 1 & 0 & 0 & 0 \\
    1 & 0 & 0 & 1 & 1 \\
    1 & 1 & 1 & 1 & 0 \\
\end{tabular}

\subsection{}

$x.z = (x+y).(x+\overline{y}).(\overline{x}+z)$

\begin{tabular}{|c|c|c||x|c|c|c|x|c|}
    $x$ & $y$ & $z$ & $x.z$ & $(x+y)$ & $.$ & $(x+\overline{y})$ & $.$ & $(\overline{x}+z)$ \\
    \hline
    0 & 0 & 0 & 0 & 0 & 0 & 1 & 0 & 1 \\
    0 & 0 & 1 & 0 & 0 & 0 & 1 & 0 & 1 \\
    0 & 1 & 0 & 0 & 1 & 0 & 0 & 0 & 1 \\
    0 & 1 & 1 & 0 & 1 & 0 & 0 & 0 & 1 \\
    1 & 0 & 0 & 0 & 1 & 1 & 1 & 0 & 0 \\
    1 & 0 & 1 & 1 & 1 & 1 & 1 & 1 & 1 \\
    1 & 1 & 0 & 0 & 1 & 1 & 1 & 0 & 0 \\
    1 & 1 & 1 & 1 & 1 & 1 & 1 & 1 & 1 \\
\end{tabular}

\section{Ejercicio 2}

Resolviendo mediante propiedades llegamos a 2 fórmulas que a priori no parecen ser equivalentes. Notar en la última línea que a la izquierda tenemos $\overline{y.z}$ mientras que a la derecha tenemos $\overline{y}.\overline{z}$.
\begin{flalign*}
x \xor (y.z) &= (x \xor y).(x \xor z) &&\\
\overline{x}.y.z + x.\overline{y.z} &= (\overline{x}.y + x.\overline{y}).(\overline{x}.z + x.\overline{z}) &&\\
\overline{x}.y.z + x.\overline{y.z} &= (\overline{x}.y + x.\overline{y}).\overline{x}.z + (\overline{x}.y + x.\overline{y}).x.\overline{z} &&\\
\overline{x}.y.z + x.\overline{y.z} &= \overline{x}.y.\overline{x}.z + x.\overline{y}.\overline{x}.z + \overline{x}.y.x.\overline{z} + x.\overline{y}.x.\overline{z} &&\\
\overline{x}.y.z + x.\overline{y.z} &= \overline{x}.y.z + x.\overline{y}.\overline{z} &&\\
\end{flalign*}

Calculamos la tabla de verdad para verificar.

\begin{tabular}{|c|c|c||c|x|c|c|x|c|c|}
    $x$ & $y$ & $z$ & $x$ & $\xor$ & $(y.z)$ & $(x \xor y)$ & . & $(x \xor z)$ & \\
    \hline
    1 & 1 & 1 & 1 & 0 & 1 & 0 & 0 & 0 &  \\
    1 & 1 & 0 & 1 & 1 & 0 & 0 & 0 & 1 & \xmark \\
    1 & 0 & 1 & 1 & 1 & 0 & 1 & 0 & 0 & \xmark \\
    1 & 0 & 0 & 1 & 1 & 0 & 1 & 1 & 1 &  \\
    0 & 1 & 1 & 0 & 1 & 1 & 1 & 1 & 1 &  \\
    0 & 1 & 0 & 0 & 0 & 0 & 1 & 0 & 0 &  \\
    0 & 0 & 1 & 0 & 0 & 0 & 0 & 0 & 1 &  \\
    0 & 0 & 0 & 0 & 0 & 0 & 0 & 0 & 0 &  \\
\end{tabular}

Conclusión: la propiedad planteada es falsa.

\section{Ejercicio 3}

\subsection{}

Verdadero, con el operador NAND ($p|q = \overline{p.q}$) podemos representar todas las funciones booleanas: AND, OR, NOT.

Recordemos la tabla de verdad del NAND.

\begin{tabular}{|c|c|c|x|}
    $p$ & $q$ & $p.q$ & $p|q = \overline{p.q}$ \\
    \hline
    0 & 0 & 0 & 1 \\
    0 & 1 & 0 & 1 \\
    1 & 0 & 0 & 1 \\
    1 & 1 & 1 & 0 \\
\end{tabular}

\subsubsection{NOT}

Utilizando la misma entrada 2 veces en un NAND podemos obtener un NOT.

$p|p = \overline{p}$

\begin{tabular}{|c|x|x|}
    $p$ & $p|p$ & $\overline{p}$ \\
    \hline
    0 & 1 & 1 \\
    1 & 0 & 0 \\
\end{tabular}

\begin{figure}[H]
    \includesvg{circuitos/NAND_NOT}
\end{figure}

\subsubsection{AND}

Utilizando el NOT ya construido, podemos encadenarlo a la salida de un NAND para cancelar su negación y así obtener el resultado original del AND.

$(p|q)|(p|q) = p.q$

\begin{tabular}{|c|c|c|x|x|}
    $p$ & $q$ & $p|p$ & $(p|q)|(p|q)$ & $p.q$ \\
    \hline
    0 & 0 & 1 & 0 & 0 \\
    0 & 1 & 1 & 0 & 0 \\
    1 & 0 & 1 & 0 & 0 \\
    1 & 1 & 0 & 1 & 1 \\
\end{tabular}

\begin{figure}[H]
    \includesvg{circuitos/NAND_AND}
\end{figure}

\subsubsection{OR}

$(p|p)|(q|q) = \overline{p}|\overline{q} = \overline{\overline{p}.\overline{q}} = \overline{\overline{p}}+\overline{\overline{q}} = p+q$

\begin{tabular}{|c|c|c|x|c|x|}
    $p$ & $q$ & $(p|p)$ & $|$ & $(q|q)$ & $p+q$ \\
    \hline
    0 & 0 & 1 & 0 & 1 & 0 \\
    0 & 1 & 1 & 1 & 0 & 1 \\
    1 & 0 & 0 & 1 & 1 & 1 \\
    1 & 1 & 0 & 1 & 0 & 1 \\
\end{tabular}

\begin{figure}[H]
    \includesvg{circuitos/NAND_OR}
\end{figure}

\end{document}
