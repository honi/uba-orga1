\documentclass{article}

\usepackage[utf8]{inputenc}
\usepackage[spanish]{babel} % Agregar es-nodecimaldot si hay problemas con el separador de decimales y de miles.
\usepackage{amsmath} % Agregar cuando se necesiten: amscd, amssymb, amsthm, latexsym
\usepackage{enumerate}
\usepackage{graphicx}
\usepackage{multicol}
\usepackage{svg}

% Tamaño de hoja y márgenes
\usepackage[a4paper,top=1cm,bottom=2cm,left=1cm,right=1cm]{geometry}
\setlength{\parindent}{0em}
\setlength{\parskip}{1em}
\setlength{\intextsep}{1em}
\setlength{\abovedisplayskip}{1em}
\setlength{\belowdisplayskip}{1em}
\setlength{\abovedisplayshortskip}{1em}
\setlength{\belowdisplayshortskip}{1em}

% Numeración de las secciones y el índice
\setcounter{tocdepth}{1}
\renewcommand{\thesubsection}{\thesection.\alph{subsection}}

% Color de los links externos
\usepackage{hyperref}
\hypersetup{colorlinks=true, linkcolor=black, urlcolor=blue}

% Permite math mode adentro de listings (bloques de "código")
\usepackage{listings}
\lstset{basicstyle=\ttfamily, mathescape}

% Alias para íconos/símbolos
\usepackage{pifont}
\newcommand{\xmark}{\color{purple}\ding{54}}

% Otros alias
\newcommand{\xor}{\oplus}
\newcommand{\nor}{\downarrow}

% Columna "x" tiene fondo amarillo
\usepackage{colortbl}
\newcolumntype{x}{>{\columncolor[HTML]{FFF2CC}}c}


\title{Organización del Computador I}
\author{Práctica 3: Arquitectura del CPU}
\date{1er cuatrimestre 2022}

\begin{document}

\maketitle
\tableofcontents
\newpage

\section{Ejercicio 1}

\subsection{Pseudocódigo: leftShift}

\begin{lstlisting}
while posiciones > 0
    valor = valor + valor
    posiciones = posiciones - 1
endwhile
\end{lstlisting}

\subsection{Assembler: leftShift}

\begin{lstlisting}
; R0 = Valor a shiftear.
; R1 = Cantidad de posiciones a shiftear.
ciclo: SUB R1, 0x0001 ; Restamos 1 al contador de posiciones R1.
       JNEG fin       ; Si R1 es <= 0 terminamos.
       ADD R0, R0     ; Sumarse a sí mismo es lo mismo que valor * 2.
       JMP ciclo      ; Saltamos a la etiqueta ciclo.
fin:   RET            ; Retornamos de la subrutina.
\end{lstlisting}

\subsection{Otros registros}

La rutina propuesta no altera los valores de los otros registros.

\section{Ejercicio 2}

Pedimos como precondición que la longitud del vector sea al menos 1.

\subsection{Pseudocódigo: minMax}

\begin{lstlisting}
max = vector[0]
min = vector[0]
i = 0
while i < longitud(vector)
    if vector[i] > max then
        max = vector[i]
    endif
    if vector[i] < min then
        min = vector[i]
    endif
    i = i + 1
endwhile
\end{lstlisting}

\pagebreak

\subsection{Assembler: minMax}

\begin{lstlisting}
; R0 = Posición de inicio del vector.
; R1 = Longitud del vector.
; R2 = Valor del máximo.
; R3 = Valor del mínimo.
main:     MOV R2, [R0]   ; Asignamos el primer elemento del vector como el máximo.
          MOV R3, [R0]   ; Asignamos el primer elemento del vector como el mínimo.
checkMax: CMP R2, [R0]   ; Comparamos el elemento del vector con el máximo.
          JGE checkMin   ; Si es <= que el máximo saltamos.
          MOV R2, [R0]   ; Guardamos el nuevo máximo.
checkMin: CMP R3, [R0]   ; Comparamos el elemento del vector con el mínimo.
          JLE next       ; Si es >= que el mínimo saltamos.
          MOV R3, [R0]   ; Guardamos el nuevo mínimo.
next:     ADD R0, 0x0001 ; Avanzamos al siguiente elemento del vector.
          SUB R1, 0x0001 ; Restamos 1 a la longitud.
          JNE checkMax   ; Si aún no llegamos a 0 repetimos el ciclo.
fin:      RET            ; Retornamos de la subrutina.
\end{lstlisting}

\section{Ejercicio 3}

\subsection{Pseudocódigo: sumar64}

El algoritmo para sumar en complemento a 2 es sumar bit a bit (no importa el total de bits). Por lo tanto, si la ALU de ORGA1 opera con palabras de 16 bits a la vez, tendremos que hacer en total 4 sumas: sumamos la primer palabra (los primeros 16 bits), luego la segunda palabra, y así sucesivamente. La primer suma la hacemos con ADD, y las otras 3 con ADDC para contemplar el carry de la palabra anterior. Al realizar la operación de esta forma, el resultado final va a ser correcto, pero los flags de la ALU no sirven, ya que solo van a indicar lo sucedido con la suma de la última palabra.

\subsection{Assembler: sumar64}

\begin{lstlisting}
; R0 = Posición del primer número de 64bits a sumar (lo llamamos A).
; R1 = Posición del segundo número de 64bits a sumar (lo llamamos B).
; R2 = Posición donde guardar el resultado.

; Sumamos la primer palabra de A con la primera de B.
; Ya lo guardamos en la primer palabra del resultado.
MOV  [R2], [R0]
ADD  [R2], [R1]
; Sumamos la segunda palabra de A con la segunda de B.
; Utilizamos el modo de direccionamiento indexado para obtener la palabra deseada
; dentro de los 64 bits. No podemos hacer ninguna cuenta en la ALU ya que perderíamos
; el flag de carry de la suma anterior.
MOV  [R2 + 0x0001], [R0 + 0x0001]
ADDC [R2 + 0x0001], [R1 + 0x0001]
; Sumamos la tercer palabra de A con la tercera de B.
MOV  [R2 + 0x0002], [R0 + 0x0002]
ADDC [R2 + 0x0002], [R1 + 0x0002]
; Sumamos la cuarta palabra de A con la cuarta de B.
MOV  [R2 + 0x0003], [R0 + 0x0003]
ADDC [R2 + 0x0003], [R1 + 0x0003]
; Retornamos de la subrutina.
RET
\end{lstlisting}

\section{Ejercicio 4}

\subsection{Pseudocódigo: sumarVector64}

\begin{lstlisting}
resultado = 0
i = 0
while i < longitud(vector)
    resultado = resultado + vector[i]
    i = i + 1
endwhile
\end{lstlisting}

\subsection{Assembler: sumarVector64}

\begin{lstlisting}
; R0 = Longitud del vector.
; R1 = Posición de inicio del vector.
; R3 = Posición donde guardar el resultado.
; R4 = Cantidad de elementos a sumar.

       ; Inicializamos el resultado en 0.
main:  MOV [R3], 0x0000
       MOV [R3 + 0x0001], 0x0000
       MOV [R3 + 0x0002], 0x0000
       MOV [R3 + 0x0003], 0x0000
       ; La cantidad de elementos a sumar es inicialmente la longitud del vector.
       MOV R4, R0
       ; Acomodamos los registros para que funcione la subrutina sumar64.
       ; Movemos la posición donde guardar el resultado a R2.
       ; También la movemos a R0 para que sumar64 haga la suma "in place".
       MOV R2, R3
       MOV R0, R3

       ; En cada ciclo restamos 1 de la cantidad de elementos a sumar (R4).
       ; Si la cantidad es <= 0 terminamos.
ciclo: SUB R4, 0x0001
       JNEG fin
       ; Invocamos la subrutina sumar64.
       ; El efecto será: [R2] = [R0] + [R1].
       ; Pero recordemos que R2 = R0 = R3, por lo tanto el efecto será: [R3] = [R3] + [R1].
       ; Y en [R1] tenemos el elemento actual del vector que queremos sumar al resultado.
       CALL sumar64
       ; Avanzamos R1 para que apunte a la posición del siguiente elemento.
       ; Hay que sumarle 4 palabras de 16 bits ya que los elementos ocupan 64 bits.
       ADD R1, 0x0004
       ; Repetimos el ciclo.
       JMP ciclo

fin:   RET
\end{lstlisting}

\section{Ejercicio 5}

Programa en assembler.

\begin{lstlisting}
; R0 = x
; R1 = y
MOV R0, 0x0002
MOV R1, 0x0020
ADD R0, R1
\end{lstlisting}

Programa ensamblado para la máquina ORGA1.

\begin{table}[ht]
\ttfamily\small
\begin{tabular}{|l|c|c|c|c|c|l|}
    Assembler & Cod. Op. & Destino & Fuente & Constante 1 & Constante 2 & Hex \\
    \hline
    MOV R0, 0x0002 & 0001 & 100000 & 000000 & 0000 0000 0000 0010 & - & 0x1800 0x0002 \\
    MOV R1, 0x0020 & 0001 & 100001 & 000000 & 0000 0000 0010 0000 & - & 0x1840 0x0020 \\
    ADD R0, R1 & 0010 & 100000 & 100001 & - & - & 0x2821 \\
\end{tabular}
\end{table}

\section{Ejercicio 6}

\begin{table}[ht]
\ttfamily\small
\begin{tabular}{|c|c|c|c|c|l|}
    Cod. Op. & Destino & Fuente & Constante 1 & Constante 2 & Assembler \\
    \hline
    0001 & 100000 & 000000 & 0000 0000 1111 1111 & - & MOV R0, 0x00FF \\
    0001 & 100001 & 000000 & 0001 0000 0000 0000 & - & MOV R1, 0x1000 \\
    0010 & 100000 & 100001 & - & - & ADD R0, R1 \\
\end{tabular}
\end{table}

\section{Ejercicio 7}

Se asume que el \lstinline{20} del enunciado es \lstinline{0x0020}.

\begin{enumerate}[a)]
    \ttfamily\small
    \item MOV R1, 0x0020 $\equiv$ R1 = 0x0020
    \item MOV R1, [0x0020] $\equiv$ R1 = [0x0020] = 0x0040
    \item MOV R1, [[0x0020]] $\equiv$ R1 = [[0x0020]] = [0x0040] = 0x0060
    \item MOV R1, R0 $\equiv$ R1 = 0x0030
    \item MOV R1, [R0] $\equiv$ R1 = [0x0030] = 0x0050
    \item MOV R1, [R0 + 0x0020] $\equiv$ R1 = [0x0030 + 0x0020] = [0x0050] = 0x0070
\end{enumerate}

\section{Ejercicio 8}

\subsection{}

\end{document}
