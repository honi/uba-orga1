\input{Algo1Macros}
\usepackage{caratula}
\usepackage{enumerate}
\usepackage{hyperref}
\usepackage{xcolor,colortbl}
\usepackage{listings}
\usepackage{multicol}

\decimalpoint
\setlength{\parindent}{0em}
\setlength{\parskip}{0.5em}
\hypersetup{colorlinks=true, linkcolor=black, urlcolor=blue}
\setcounter{tocdepth}{3}
\lstset{basicstyle=\ttfamily, mathescape}
\newcolumntype{x}{>{\columncolor[HTML]{FFF2CC}}c}
\renewcommand{\thesubsection}{\thesection.\alph{subsection}}

\begin{document}

\titulo{Práctica 1}
\subtitulo{Representación de la información}
\fecha{1er cuatrimestre 2022}
\materia{Organización del Computador 1}
\integrante{Jonathan Bekenstein}{348/11}{jbekenstein@dc.uba.ar}

\maketitle
\tableofcontents
\newpage

\section{Ejercicio 1}

\subsection{}

\begin{multicols}{3}

\begin{lstlisting}
33 = 16 $\times$ 2 + 1
16 =  8 $\times$ 2 + 0
 8 =  4 $\times$ 2 + 0
 4 =  2 $\times$ 2 + 0
 2 =  0 $\times$ 2 + 1
$\implies$ 33$_{10}$ = 10001$_2$
\end{lstlisting}

\begin{lstlisting}
33 = 11 $\times$ 3 + 0
11 =  3 $\times$ 3 + 2
 3 =  1 $\times$ 3 + 0
 1 =  0 $\times$ 3 + 1
$\implies$ 33$_{10}$ = 1020$_3$
\end{lstlisting}

\begin{lstlisting}
33 = 6 $\times$ 5 + 3
 6 = 1 $\times$ 5 + 1
 1 = 0 $\times$ 5 + 1
$\implies$ 33$_{10}$ = 113$_5$
\end{lstlisting}

\vspace*{\fill}
\columnbreak

\begin{lstlisting}
100 = 50 $\times$ 2 + 0
 50 = 25 $\times$ 2 + 0
 25 = 12 $\times$ 2 + 1
 12 =  6 $\times$ 2 + 0
  6 =  3 $\times$ 2 + 0
  3 =  1 $\times$ 2 + 1
  1 =  0 $\times$ 2 + 1
$\implies$ 100$_{10}$ = 1100100$_2$
\end{lstlisting}

\begin{lstlisting}
100 = 33 $\times$ 3 + 1
 33 = 11 $\times$ 3 + 0
 11 =  3 $\times$ 3 + 2
  3 =  1 $\times$ 3 + 0
  1 =  0 $\times$ 3 + 1
$\implies$ 100$_{10}$ = 10201$_3$
\end{lstlisting}

\begin{lstlisting}
100 = 20 $\times$ 5 + 0
 20 =  4 $\times$ 5 + 0
  4 =  0 $\times$ 5 + 4
$\implies$ 100$_{10}$ = 400$_5$
\end{lstlisting}

\vspace*{\fill}
\columnbreak

\begin{lstlisting}
1023 = 511 $\times$ 2 + 1
 511 = 255 $\times$ 2 + 1
 255 = 127 $\times$ 2 + 1
 127 =  63 $\times$ 2 + 1
  63 =  31 $\times$ 2 + 1
  31 =  15 $\times$ 2 + 1
  15 =   7 $\times$ 2 + 1
   7 =   3 $\times$ 2 + 1
   3 =   1 $\times$ 2 + 1
   1 =   0 $\times$ 2 + 1
$\implies$ 1023$_{10}$ = 1111111111$_2$
\end{lstlisting}

\begin{lstlisting}
1023 = 341 $\times$ 3 + 0
 341 = 113 $\times$ 3 + 2
 113 =  37 $\times$ 3 + 2
  37 =  12 $\times$ 3 + 1
  12 =   4 $\times$ 3 + 0
   4 =   1 $\times$ 3 + 1
   1 =   0 $\times$ 3 + 1
$\implies$ 1023$_{10}$ = 1101220$_3$
\end{lstlisting}

\begin{lstlisting}
1023 = 204 $\times$ 5 + 3
 204 =  40 $\times$ 5 + 4
  40 =   8 $\times$ 5 + 0
   8 =   1 $\times$ 5 + 3
   1 =   0 $\times$ 5 + 1
$\implies$ 1023$_{10}$ = 13043$_5$
\end{lstlisting}

\end{multicols}

\subsection{}

\emph{Nota: a la derecha del igual se interpretan todos los números en base 10.}

\begin{lstlisting}
1111$_2$ = (1 $\times$ 2$^3$) + (1 $\times$ 2$^2$) + (1 $\times$ 2$^1$) + (1 $\times$ 2${^0}$)
= 8 + 4 + 2 + 1
= 15
\end{lstlisting}

\begin{lstlisting}
1111$_3$ = (1 $\times$ 3$^3$) + (1 $\times$ 3$^2$) + (1 $\times$ 3$^1$) + (1 $\times$ 3${^0}$)
= 27 + 9 + 3 + 1
= 40
\end{lstlisting}

\begin{lstlisting}
1111$_5$ = (1 $\times$ 5$^3$) + (1 $\times$ 5$^2$) + (1 $\times$ 5$^1$) + (1 $\times$ 5${^0}$)
= 125 + 25 + 5 + 1
= 156
\end{lstlisting}

\begin{lstlisting}
CAFE$_{16}$ = (12 $\times$ 16$^3$) + (10 $\times$ 16$^2$) + (15 $\times$ 16$^1$) + (14 $\times$ 16${^0}$)
= 49152 + 2560 + 240 + 14
= 51966
\end{lstlisting}

\pagebreak
\subsection{}

\begin{lstlisting}
17$_8$ = (1 $\times$ 8$^1$) + (7 $\times$ 8$^0$) = (15)$_{10}$

15 = 3 $\times$ 5 + 0
 3 = 0 $\times$ 5 + 3

$\implies$ 17$_8$ = 15$_{10}$ = 30$_5$
\end{lstlisting}

\begin{lstlisting}
BABA$_{13}$ = (11 $\times$ 13$^3$) + (10 $\times$ 13$^2$) + (11 $\times$ 13$^1$) + (10 $\times$ 13$^0$) = (26010)$_{10}$

26010 = 4335 $\times$ 6 + 0
 4335 =  722 $\times$ 6 + 3
  722 =  120 $\times$ 6 + 2
  120 =   20 $\times$ 6 + 0
   20 =    3 $\times$ 6 + 2
    3 =    0 $\times$ 6 + 3

$\implies$ BABA$_{13}$ = 26010$_{10}$ = 320230$_6$
\end{lstlisting}

\subsection{}

\begin{lstlisting}
(10 01 01 10 10 10 01 01)$_2$ = 21122211$_4$
(001 001 011 010 100 101)$_2$ = 113245$_8$
(1001 0110 1010 0101)$_2$ = 96A5$_{16}$
\end{lstlisting}

\section{Ejercicio 2}

\end{document}
